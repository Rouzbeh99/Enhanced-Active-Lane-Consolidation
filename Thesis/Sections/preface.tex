\begin{preface}

\setlength{\baselineskip}{1.6\baselineskip} 
An early version of the content in Chapter \ref{chap:background}-\ref{chap:evaluation} of this thesis was submitted as R. Paktinatkeleshteri, J. P. L. De Carvalho, E. Amiri and J. N. Amaral "Efficient Auto-Vectorization for Control-flow Dependent Loops through Data Permutation" to CASCON 2023.
The original idea of ALC was proposed by Praharenka \etal. My role was to develop ALC as a compiler pass to apply it automatically to programs, to evaluate its performance on real hardware and to propose transformations that were necessary in order for ALC to deliver performance in actual hardware.  J. P. L. Carvalho made valuable by contributions by offering recommendations in technical discussions and guiding the direction of the project. E. Amiri contributed  by providing essential infrastructure support and engaging in technical discussions. J.N. Amaral supervised all aspects of the project, guiding the experimental methodology and enhancing the resulting manuscript.

This thesis extends the submitted paper with an in-depth discussion of ALC and the improvements proposed in this work (Chapter \ref{chap:alc-pass}).
Moreover, the evaluation in this thesis provides evidence of the broad applicability of ALC, which was not part of the paper submitted to CASCON 2023.

\end{preface}